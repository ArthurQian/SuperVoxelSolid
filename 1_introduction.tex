\section{Introduction}
Solid texture plays an essential role in virtual internal material modeling and subsurface scattering effects simulation. However, acquiring a solid texture is much harder than that of 2D texture. Moreover, high memory cost for storage also limits its usage. Synthesis methods provide an alternative way to achieve solid texture. 

Procedural noises \cite{perlin1985image} model solid texture by procedural function. Pixel color can be efficiently accessed at runtime without pre-synthesis. It's popular for its compelling properties like resolution independent representation and straightforward implementation. It's however limited in restricted patterns and hard to tune for specific appearance. Mush efforts have been made to generate solid texture from one or several input exemplars \cite{kopf2007solid,du2013semiregular}. Most of the methods focus on generating raster solids which acquire high memory storage for synthetic results. A few works \cite{zhang2013efficient,shu2014efficient} directly generate vector solid texture by introducing vector representations, such as vector solid particles and gradient solids. The methods are either limited in specific solids or the capability to maintain example macro-structures. \revision{The Limitations.}

\revision{Our objective and target texture type.} Our objective is to generate solid texture from example with several attractive properties: compact representation, generality. Inspired by super-pixel of image which segments image into homogeneous regions with similar color, we propose super-voxel solid texture represented by a set of super-voxel elements. Each element consists in a center position, a radial basis function to represent color variation and a signed distance function to model region boundaries. \revision{Pixel evaluation.} By observing the intuitive similarity between image super-pixel segmentation and super-voxel representation, we design synthesis algorithm with texture optimization framework for discrete element arrangement. The framework is designed based on super-voxel elements, rather than traditional raster voxels. Thus it owns the desired properties. 

The main contribution of this paper are as follows:
\begin{itemize}
	\item Propose super-voxel solid texture model which has compact and resolution independent representation. A powerful extension provides more generality and capability for thin features. \revision{Vector representation.}
	\item Design vector solid synthesis algorithm based on texture optimization framework by observing the similarity between image super-pixel segmentation and our super-voxel solid model.
\end{itemize}