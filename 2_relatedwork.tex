\section{Related Work}
\label{sec:relarted}

Our objective is to synthesize vector solid texture represented by discrete super-voxel elements from 2D examples. Thus we review most related work in three categories: solid texture synthesis, vector texture modeling and texture element arrangement. Comprehensive survey for solid texture synthesis and example based texture synthesis can refer to \cite{pietroni2010solid} and \cite{wei2009state}, respectively.

\noindent
\textbf{Example-based Solid Texture Synthesis}
There are plenty of work regenerating solid textures from 2D examples.
Heeger \etal \cite{heeger1995pyramid} synthesize stochastic solid texture by matching distribution of filter outputs. Image pyramid is introduced to retain stochastic in different scales. 
Dischler \etal \cite{dischler1998anisotropic} further introduces spectral matching to reproduce solid textures from orthogonal 2D view exemplars. 
Wei \cite{wei2003texture} extends non-parametric 2D texture synthesis algorithm \cite{wei2000fast} to synthesize solid texture with multiple input exemplars.
Kopf \etal \cite{kopf2007solid} synthesize solid texture based on texture optimization framework \cite{kwatra2005texture}. Histogram matching is adopted for fast convergence.
Dong \etal \cite{dong2008lazy} propose an on-demand synthesis algorithm by reducing the search space with a carefully computed candidate set which can form consistent triples. Zhang \etal \cite{zhang2011sketch} design a deterministic synthesis algorithm which can generate textures agreeing with a tensor field derived from user sketch curves. A historical window representation is proposed to guarantee the effectiveness of correction scheme. Zhang \etal \cite{zhang2013efficient} propose a vector representation called gradient solid texture. Control points with assigned feature vector on uniform grid is introduced to compactly represent solid texture. Texture optimization scheme is designed by defining energy function between the control points and 2D example.

Some synthesis algorithm are designed for specific solid texture patterns. 
Jagnow \etal \cite{jagnow2004stereological} generate aggregate solid texture by distributing manually designed particles into target volume. Stereology techniques are adopted to estimate 3D particle distribution from 2D cross sections.
Takayama \etal \cite{takayama2008lapped} propose an interactive method to lap solid texture patches in specified domain. Nonhomogeneous solid texture can be generated with a depth field indicating texture pattern variation. Du \etal \cite{du2013semiregular} reproduce solid texture with semi-regular pattern by placing reconstructed particles with the guidance of 2D particle neighbor information from input example. 3D particles is reconstructed from triples of 2D cross-sections with a morphing scheme.  Shu \etal \cite{shu2014efficient} focus on effectively generating aggregate solid texture with automatically reconstructed particles from 2D cross section. A compact particle model is introduced to support vector solid texture representation. Qian \etal \cite{qian2015vector} propose a two-scale shaping model to compactly represent particles with much details. Palacios \etal \cite{palacios2016tensor} propose a 3D tensor field design system and apply it to generate solid textures aligned with object geometry features.

Most of methods generate raster solid textures. Even our algorithm shares a similar idea to that of Zhang \etal \cite{zhang2013efficient}, our synthesis algorithm first convert a raster example image into vector 2D representation and then conduct synthesis process based on the vector primitives in 2D example and target solid texture.

\noindent
\textbf{Vector Solid Textures}
Vector graphics have some advantages prior to raster images, such as compact storage and easy to edit. These advantages greatly make up deficiencies of solid texture. Takayama \etal \cite{takayama2010volumetric} propose diffusion surfaces to model volumetric objects with internal structures. Colors are defined on both sides of the surfaces. A modified version of the positive mean value coordinates is proposed to interpolate colors in space locally without volumetric meshing. Wang \etal \cite{wang2010vector} introduce a vector representation for solid texture with intermixed regions. Signed distance field is used to represent region boundaries, and radial basis functions is used to represent color variation in each region. Algorithm is designed to convert raster solid texture to the vector representation. To represent volume with complex internal structure spanning a wide range of scales, Wang \etal \cite{wang2011multiscale} decompose an object into components and model each component as a binary SDF tree. Multiscale embedding is used to combine object components at different scales. Zhang \etal \cite{zhang2013efficient} represent vector solid texture by a uniform grid with  each control point assigned with a feature vector. The feature vector consists color components, signed distance and corresponding gradients. Shu \etal \cite{shu2014efficient} and Qian \etal \cite{qian2015vector} propose vector representation for vector aggregate textures. Point based particle and signed distance particle are introduced to represent particles respectively. Most of the methods focus on vector representation only, rather than our objective, \ie synthesizing vector solid texture from a raster 2D example.

\noindent
\textbf{Texture Element Arrangement}
There are a variety of methods focusing on generating element placement with guidance of example. Dischler \etal \cite{dischler2002texture} propose texture particle which decompose 2D texture into elementary components. Correspondingly, synthesis process recompose similar texture by taking into account example spatial arrangements. It's further generalized to surface texture. Ijira \etal \cite{ijiri2008example} generate 2D element arrangement based on local growth combing local neighbor comparison and procedural modeling systems. Global feature can be introduced by user interaction. Hurtut \etal \cite{hurtut2009appearance} examine example elements shape and divide the elements into categories. Arrangement statistical model is estimated to capture spatial interactions within, and between categories. New arrangements are generated by Monte-Carlo Markov chain sampling. Ma \etal \cite{ma2011discrete} synthesize repetitive elements according to a small input example within a large output domain. Optimization algorithm is designed based on distance metric defined on discrete element neighbor. It's further extended to dynamic motions\cite{ma2013dynamic} by introducing a temporal dimension.

Our synthesis algorithm assembles the elements into target volume and adjust the distribution with guidance of example super-pixel distribution. It's similar to texture element arrangement algorithms. However, the main difference is that we produce 3D discrete distribution from 2D example, rather than conducting arrangement on the same dimension. Specific distance measurement need to be designed for the optimization algorithm. 